
\chapter{結論} \label{conclusion_and_future}

\section{命令總結}

\subsection*{產生論文}

經過開發者討論後,決議將論文與封面產生皆由 all 指令產生。

\begin{lstlisting}[language=bash]
    make all
\end{lstlisting}

\subsection*{清除暫存}

暫存檔案檔案是由 xelatex 編譯時所產生,使用此指令清除暫存並不會將 PDF 檔案清除。

\begin{lstlisting}[language=bash]
    make clean
\end{lstlisting}

\subsection*{完整清除}

完整清除檔案會將所有由 xelatex 編譯產生的檔案完整清除,包含論文與封面的 PDF 檔案。

\begin{lstlisting}[language=bash]
    make distclean
\end{lstlisting}

\subsection*{PDF防拷加工}

透過 ghostscript\cite{ghostscript} 為 PDF 提供防拷功能,其指令如下。

\begin{lstlisting}[language=bash]
    make pdfprocessing
\end{lstlisting}

此功能依賴 build/main.pdf 作為加工來源,因此當 build/main.pdf 不存在導致無法執行此命令時,請先執行 make all 來產生檔案。

\section{結語}

未來,我們也不知道能持續維護多久,只要有人使用,我們就會盡力維護下去。
目前僅支援 Linux,未來將會盡量朝向支援 Windows,讓更多人能使用此專案。

如果這個專案給您提供了幫助,請給我們 GitHub Repository\cite{github-repo} 一個 Star 讓我們知道。
最後祝福您順利畢業!
