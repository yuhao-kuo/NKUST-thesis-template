
\chapter{Conclusions and Future Work} \label{conclusion_and_future}
In the Introduction, we expressed the hope that we could made three distinctive contributions toward WSN, WRSN and FRID network, respectively. In this Chapter, we will conclude by describing the proposed method process.
We will also suggest some future work that could provide general view about proposed practically applications
in this thesis.   

\section{Conclusions}

Firstly, in Chapter \ref{chapter1}, we investigated the weighted-critical-square-grid
coverage problem, which is the problem of using limited sensors to
construct a wireless sensor network such that the total weight of
the covered critical square grids is maximized. The problem was
shown to be NP-complete. In addition, a reduction that transforms
the weighted-critical-square-grid coverage problem into the
constrained node-weighted Steiner tree problem was proposed. Once a
solution to the constrained node-weighted Steiner tree problem is
obtained, the solution can be used to select the points that are
allowed to deploy sensors for the weighted-critical-square-grid
coverage problem. We also showed that the constrained node-weighted
Steiner tree problem is NP-complete. In addition, the greedy
algorithm (GA), the group-based algorithm (GBA), and the
profit-based algorithm (PBA), were proposed for the constrained
node-weighted Steiner tree problem.

In Chapter \ref{chapter2}, we then presented the problem of scheduling minimum mobile
devices to periodically recharge and collect data from sensors
subject to the limited charging range, electric capacity, and memory
storage, such that the network lifetime can be guaranteed to be
prolonged without limits, termed the Periodic Energy Replenishment
and Data Collection (PERDC) problem. We first showed
that the PERDC problem is NP-complete. In addition, three
algorithms, including the grid-based algorithm (GBA), the
dominating-set-based algorithm (DSBA), and the
circle-intersection-based algorithm (CIBA), were proposed to find a
set of anchor points. Based on the generated anchor points, the
mobile device scheduling algorithm (MDSA) was proposed to schedule
minimum mobile devices for energy replenishment and data collection.
Moreover, theoretical analysis showed that the GBA had the lowest
time complexity compared with the DSBA and the CIBA when the side
length of the sensor field and the sensor's charging range are
constant. In the future works, we wish to evaluate some metrics
of the network to show the performance of our proposed via extensive simulations.

Finally, in Chapter \ref{chapter3}, we introduced 
the problem of activating readers and
adjusting their interrogation ranges to cover maximum tags without
collisions, termed the Reader-Coverage Collision Avoidance
Arrangement (RCCAA) problem. In order to show the difficulty of the
RCCAA problem, a sub-problem of the RCCAA problem, termed the
Reduced Reader-Coverage Collision Avoidance Arrangement (RRCCAA)
problem, was shown to be NP-hard. In addition, an approximation
algorithm, termed the Maximum-Weight-Independent-Set-Based Algorithm
(MWISBA), was proposed for the RCCAA problem. We also provided
theoretical analysis of our proposed method in terms of the
correctness, the time complexity, and the performance guarantee.
To evaluate our performance, a method that extended from the greedy
distributed elimination (GDE) method \cite{related1}, called the
Extended Greedy Distributed Elimination Algorithm (EGDEA), was
proposed for the RCCAA problem. The MWISBA was compared with the
EGDEA in terms of the number of tags read by readers and the number
of activated readers. The simulation results showed that the readers
activated by the MWISBA can read a significantly higher number of
tags than that in the EGDEA.

\section{Suggestions for Future Work} \label{futurework}

Nowadays, a network model is not simply a static network model when both hardware
and software of network system has been significantly
improved. The designing and proposition of an optimum
solution for network system therefore has
become complicated as well. For example, network
architecture keeps changing constantly and it is divided
into layers with different levels of importance
rather than having static nature as before. More
complicated network models require new techniques
and algorithm to meet real requirements. In addition,
diversified development of functions of network
elements also means more complicated potential conflicts 
in network routing and interfacing between
network elements. To incorporate these functions
and avoid these new conflicts, new network models of
high degree of diversification are required. Based on
experience in the design and optimizing algorithm in
such network types as wireless sensor network, mobile
ad hoc network, RFID system, I will continue to
expand my researches to tackle new issues in different
network architecture.

My current task is to find solutions for developing
and optimizing algorithms in network structure development,
optimization of routing of such networks
as wireless sensor network, mobile ad hoc network,
and RFID system. Nevertheless, to complete and
make my research more independent, I shall continually
expand with new research fields as mentioned to
tackle many aforementioned issues such as complex network, interdependent network. 


