
\chapter{緒論}\label{introduction}


\section{前言}\label{preface}

NKUST LaTeX 論文版型提供給本校研究生撰寫論文。當您使用這個專案,表示您已經進入碩士生涯的最後階段,祝福您也恭喜您即將完成碩士學業。我們希望這個專案能在論文撰寫的路上給您提供助力。

本專案所使用的工具皆為 open source 軟體,可放心地由網路上自由下載合法的免費使用,文件編譯工具皆採用 TUG(TEX Users Group) 提供的 TexLive 套件包,編輯器使用 Microsoft 的 VSCode,並以 GNU Bash 環境作為開發的基礎。

目前當前我們正在測試讓這個版型可以運作於 Windows 系列的作業系統上,預計在 release v2.1 版本開始提供 Windows 上的支援。

\newpage

\section{研究動機}\label{motive}

那年的春天,我們從已經畢業的學長得知,過去的板型有一部份已經不符合學校新的規範,因此我們開始了板型重新編排的計劃。起初只是實驗室間共同開發,後來陸續有更多同學在 GitHub 上提交貢獻,也許多使用者也提供我們許多實用的建議,讓這個專案架構越來越完整,也讓整個環境往更簡便的方向前進。目前專案已經由學弟們維護,期望未來能夠幫助更多需要使用LaTeX撰寫論文的研究生。


\newpage

\section{論文架構}\label{thesis_arch}
\n 本論文編排方式如下:

第\ref{ch_tmp_config}章 版型設定

第\ref{ch_content}章 $LaTeX$ 文字範例

第\ref{algorithm}章 演算法虛擬碼範例

第\ref{Experimental_picture}章 模擬實驗與結果分析

第\ref{conclusion_and_future}章 結論
