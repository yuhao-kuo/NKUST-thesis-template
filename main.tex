%----------------------------------------------------------------------------------------
% NKUST EC Master Thesis
% Original authors : AIoRLab/NCLab/WNDCLab Thesis LaTeX Team
% Department of Electronic Engineering
% National Kaohsiung University of Sciences and Technology
% GitHub : https://github.com/yuhao-kuo/NKUST-thesis-template
% Version 1.0 (2020.01.31)
%----------------------------------------------------------------------------------------

%----------------------------------------------------------------------------------------
%	PACKAGES AND OTHER DOCUMENT CONFIGURATIONS
%----------------------------------------------------------------------------------------

\documentclass[oneside,
12pt, % The default document font size, options: 10pt, 11pt, 12pt
%oneside, % Two side (alternating margins) for binding by default, uncomment to switch to one side
english, % ngerman for German
onehalfspacing, % Single line spacing, alternatives: onehalfspacing or doublespacing
%draft, % Uncomment to enable draft mode (no pictures, no links, overfull hboxes indicated)
%nolistspacing, % If the document is onehalfspacing or doublespacing, uncomment this to set spacing in lists to single
%liststotoc, % Uncomment to add the list of figures/tables/etc to the table of contents
%toctotoc, % Uncomment to add the main table of contents to the table of contents
%parskip, % Uncomment to add space between paragraphs
]{NKUSTMastersThesis} % The class file specifying the document structure

\usepackage{CJKutf8}
\usepackage[utf8]{inputenc} % Required for inputting international characters
\usepackage[T1]{fontenc} % Output font encoding for international characters
\usepackage{palatino} % Use the Palatino font by default
\usepackage{indentfirst} \setlength{\parindent}{2em}
\usepackage[printwatermark]{xwatermark}
\usepackage{xcolor}

\usepackage{lettrine}  % used for chinese bigger capital
\usepackage{transparent}
\usepackage{slashbox,booktabs,amsmath}
\usepackage{algorithm}
\usepackage{transparent}

\usepackage{enumerate}
\usepackage{multirow}
%\usepackage[backend=bibtex,style=authoryear,natbib=true]{biblatex} % User the bibtex backend with the authoryear citation style (which resembles APA)

% \addbibresource{example.bib} % The filename of the bibliography
\frontmatter % Use roman page numbering style (i, ii, iii, iv...) for the pre-content pages
\pagestyle{plain} % Default to the plain heading style until the thesis style is called for the body content
\usepackage{makecell}
\cleardoublepage
\usepackage[autostyle=true]{csquotes} % Required to generate language-dependent quotes in the bibliography

% \newwatermark[pages=3-112,fontfamily=bch,color=gray!100,scale=4,xpos=4,ypos=20]{\transparent{0.4}\includegraphics[width=1.25cm]{\watermarkimage}}
\usepackage{pdfpages}

%
% Thesis Defined
%

% ------------ 作者資訊 ------------

\def\authortwname{王小明}  % 研究生中文名字
\def\authorenname{Shio-Min Wang}  % 研究生英文名字

\def\supervisortwname{孫思邈 博士}  % 指導教授中文名字
\def\supervisorenname{Si-Miao Sun Ph.D.}  % 指導教授英文名字

\def\schooltwname{國立高雄科技大學}
\def\schoolenname{National Kaohsiung University of Science and Technology}
\def\schoolenlocation{Kaohsiung, Taiwan, Republic of China}
\def\schoolenoldname{National Kaohsiung University of Applied Sciences} % 國立高雄科技大學由"國立高雄應用科技大學/國立高雄第一科技大學/國立高雄海洋科技大學"三所科技大學合併而成,此欄位請填入原先校區所屬學校英文名稱。

\def\majortwname{Electronic Engineering}  % 主修項目
\def\depttwname{電子工程系}  % 系所中文名稱
\def\deptenname{Department of Electronic Engineering}  % 系所英文名稱

\def\degreetw{碩士}  % 學位
\def\degreeen{Master}  % 學位

\def\titletw{高雄科技大學LaTeX論文樣板}  % 論文題目(中文)
\def\titleen{NKUST LaTeX Thesis Template}  % 論文題目(英文)

\def\dateROC{中華民國 一零九 年 六 月}  % 日期(月份)
\def\dateen{Jun, 2020}  % 英文日期


% ------------- Logo --------------

\def\thesistitlelogo{Figures/Logos/nkust.png}   % 主頁面logo
\def\watermarkimage{Figures/Logos/nkust.png}    % 浮水印


% ------------ 論文封面 ------------

\provideboolean{isthesistitleexternal}
\provideboolean{isthesisbooknameexternal}

\setboolean{isthesistitleexternal}{false}       % 封面, false: LaTeX生成, true: 匯入外部pdf
\def\externalmaintitle{Externals/maintitle}     % 外部載入封面位置, 設定由 LaTeX 生成時, 此欄位請忽視

\setboolean{isthesisbooknameexternal}{false}    % 書名頁, false: LaTeX生成, true: 匯入外部pdf
\def\externalbooktitle{Externals/booktitle}     % 外部載入書名頁位置, 設定由 LaTeX 生成時, 此欄位請忽視

\provideboolean{thesisdraft}
\setboolean{thesisdraft}{true}  % 初稿, true: 初稿, false: 非初稿


% ------------ 論文相關文件 ------------

\provideboolean{thesisauht}
\provideboolean{thesissign_zhtw}
\provideboolean{thesissign_en}
\provideboolean{thesisphdrecommand}

% 博碩士論文授權書
\def\thesispowerofattorney{Externals/powerofattorney.pdf}   % 博碩士論文授權書位置
\setboolean{thesisauht}{false}                               % 博碩士論文授權書, true: 載入, false: 不載入

% 論文口試委員會審定書
\def\thesisvalidationzhtw{Externals/sign.pdf}                   % 論文口試委員會審定書位置
\setboolean{thesissign_zhtw}{true}                               % 論文口試委員會審定書, true: 載入, false: 不載入

% 論文口試委員會英文審定書
\def\thesisvalidationen{Externals/sign_en.pdf}                   % 論文口試委員會審定書位置
\setboolean{thesissign_en}{true}                                  % 論文口試委員會審定書, true: 載入, false: 不載入

% 博士論文推薦書
\def\thesisphdrecommand{Externals/recommand.pdf}            % 博士論文推薦書位置
\setboolean{thesisphdrecommand}{false}                      % 博士論文推薦書, true: 載入, false: 不載入


% ---------- 誌謝 / 序言 ----------
\provideboolean{thesisacknowledgement}

\setboolean{thesisacknowledgement}{true}    % 誌謝 / 序言, true: 載入, false: 不載入

% ---------- 目錄 -----------
\provideboolean{thesiscontent}
\provideboolean{thesistable}
\provideboolean{thesisfiguretable}

% 目錄
\setboolean{thesiscontent}{true}        % 目錄, true: 載入, false: 不載入

% 表目錄
\setboolean{thesistable}{true}          % 表目錄, true: 載入, false: 不載入

% 圖目錄
\setboolean{thesisfiguretable}{true}    % 圖目錄, true: 載入, false: 不載入

% ---------- 附錄 -----------

\provideboolean{thesisappendix}

\setboolean{thesisappendix}{false}      % 附錄, true: 載入, false: 不載入

% --------------------------


%%----------------------------------------------------------------------------------------
%	THESIS INFORMATION
%----------------------------------------------------------------------------------------

\thesistitle{論文題目寫在這} % Your thesis title, this is used in the title and abstract, print it elsewhere with \ttitle
\supervisor{謝慶發 教授} % Your supervisor's name, this is used in the title page, print it elsewhere with \supname
\examiner{} % Your examiner's name, this is not currently used anywhere in the template, print it elsewhere with \examname
\degree{碩士} % Your degree name, this is used in the title page and abstract, print it elsewhere with \degreename
\author{郭育豪} % Your name, this is used in the title page and abstract, print it elsewhere with \authorname
\addresses{} % Your address, this is not currently used anywhere in the template, print it elsewhere with \addressname

\subject{電子工程系} % Your subject area, this is not currently used anywhere in the template, print it elsewhere with \subjectname
\keywords{} % Keywords for your thesis, this is not currently used anywhere in the template, print it elsewhere with \keywordnames
\university{國立高雄應用科技大學} % Your university's name and URL, this is used in the title page and abstract, print it elsewhere with \univname
\department{電子工程系} % Your department's name and URL, this is used in the title page and abstract, print it elsewhere with \deptname
\group{Wireless Networking and Distributed Computing Lab} % Your research group's name and URL, this is used in the title page, print it elsewhere with \groupname
%\faculty{\href{http://faculty.university.com}{Faculty Name}} % Your faculty's name and URL, this is used in the title page and abstract, print it elsewhere with \facname

\hypersetup{pdftitle=\ttitle} % Set the PDF's title to your title
\hypersetup{pdfauthor=\authorname} % Set the PDF's author to your name
\hypersetup{pdfkeywords=\keywordnames} % Set the PDF's keywords to your keywords 

\begin{document}
\begin{CJK}{UTF8}{bkai}
\CJKindent
\newcommand\n{\mbox{\qquad}}

\frontmatter % Use roman page numbering style (i, ii, iii, iv...) for the pre-content pages
\pagestyle{plain} % Default to the plain heading style until the thesis style is called for the body content


%----------------------------------------------------------------------------------------
%	論文編印項目次序
%   編排順序依照學術論文格式規範文件進行排序
%----------------------------------------------------------------------------------------

% ----- 封面頁 -----
%----------------------------------------------------------------------------------------
%	BOOK TITLE PAGE
%----------------------------------------------------------------------------------------

\begin{titlepage}
\vspace*{1mm}

\begin{center}


\includegraphics[scale=0.3]{\thesistitlelogo}\\
\vspace{10mm}
{\huge\bfseries{\schooltwname}\\
\vspace{4.5mm}
\huge\bfseries{\depttwname{ \degreetw班}}\\
\vspace{10mm}
\huge\bfseries \degreetw論文}\\
\vspace{10mm}


\vspace{10mm}

{\LARGE  \titletw}

\vspace{4.5mm}
\vspace{1\baselineskip}
{\LARGE  \titleen}
\vspace{10mm}\\

\vspace{0.5\baselineskip}
%\LARGE  {(初稿)}
\vspace{6mm}

\begin{tabular}{rl}
\large\makebox[5em][s]{研\hspace{\fill}究\hspace{\fill}生}: & \large \authortwname \\
\large\makebox[5em][s]{指導教授}: & \large\supervisortwname \\
\end{tabular}

\vspace{20mm}
{\large\textsc{\dateROC}}
\vspace{15mm}
\end{center}

\end{titlepage} 

% ----- 書名頁 -----
%----------------------------------------------------------------------------------------
%	書名頁
%----------------------------------------------------------------------------------------

\begin{titlepage}
\vspace*{1mm}

\begin{center}

{\LARGE\bfseries  \titletw}\\
\vspace{15mm}
{\LARGE  \titleen}
\vspace{15mm}

{\large\bfseries{研究生:}\large\authortwname\\
\large\bfseries{指導教授:}\large\supervisortwname}

\vspace{15mm}
{\Large\bfseries{\schooltwname}\\
\vspace{4.5mm}
\Large\bfseries{\depttwname{\degreetw 班}}\\
\vspace{4.5mm}
\Large\bfseries \degreetw 論文}\\
\vspace{10mm}

\vspace{4.5mm}
A Thesis Submitted to \deptenname\\
\schoolenname\\
in Partial Fulfillment of the Requirements\\
for the Degree of \degreeen \,of Engineering\\
in \majortwname

\vspace{15mm}
\dateen\\
\schoolenlocation

\vspace{10mm}
\schoolenoldname \,is the predecessor of
National Kaohsiung University of
Science and Technology (renamed on Feb. 1, 2018)

\vspace{10mm}
% {\large\bfseries{\dateROC}}
\fontsize{14pt}{0pt}{\bfseries{\dateROC }}

\end{center}

\end{titlepage} 

% ----- 博碩士論文授權書 -----
\includepdf[pagecommand={\thispagestyle{empty}}]{Externals/powerofattorney.pdf}

% ----- 博士論文指導教授推薦書 (碩士可直接註解省略) -----
% \includepdf[pagecommand={\thispagestyle{empty}}]{Externals/recommend.pdf}

% 論文口試委員會審定書
\includepdf[pagecommand={\thispagestyle{empty}}]{Externals/sign.pdf}

%----------------------------------------------------------------------------------------
%	論文內容 由此開始
%----------------------------------------------------------------------------------------
% 浮水印設定
\newwatermark[pages=1-100,fontfamily=bch,color=gray!100,scale=4,xpos=4,ypos=20]{\transparent{0.4}\includegraphics[width=1.25cm]{\watermarkimage}}
\setcounter{page}{1}  % 摘要為第一頁

% 中英文摘要
\renewcommand{\abstractname}{摘要}
\clearpage
\phantomsection
\addchaptertocentry{\abstractname}
\begin{cntabstract}
\n 本專案為國立高雄科技大學研究所論文 LaTex 模板,依照國立高雄科技大學學位論文格式規範及建工校區電子工程系網路計算實驗室(WNDCLab、NCLab、AIoRLab)實驗室論文格式標準進行編排。

\hbox{}
\it 關鍵詞:LaTeX、論文、模板
\end{cntabstract} \addchaptertocentry{\abstractname}
\renewcommand{\abstractname}{Abstract}
\clearpage
\phantomsection
\addchaptertocentry{\abstractname}
\begin{engabstract}

With the advancement of science and technology, people's lives are becoming more and more convenient ... the rest is left to you

\hbox{}
\it{Keywords: Artificial intelligence,Internet of Things}
\end{engabstract} \addchaptertocentry{\abstractname}

% 誌謝 or 序言
\renewcommand{\acknowledgementname}{誌謝}
\begin{acknowledgements}

謝謝天~ 謝謝地~ 謝謝蜂蜜檸檬!
\end{acknowledgements}  \addchaptertocentry{\acknowledgementname}

% 目錄
\renewcommand{\contentsname}{目錄}
\tableofcontents \addchaptertocentry{\contentsname}
\newpage

% 表目錄
\renewcommand{\tablename}{表}
\renewcommand{\listtablename}{表目錄}
\listoftables \addchaptertocentry{\listtablename}
\newpage

% 圖目錄
\renewcommand{\figurename}{圖}
\renewcommand{\listfigurename}{圖目錄}
\listoffigures \addchaptertocentry{\listfigurename}
\newpage

% 符號說明

% 論文本文
\mainmatter % Begin numeric (1,2,3...) page numbering
\ifpdf
    \graphicspath{{MyFigures/chapter1/PNG/}{MyFigures/chapter1/PDF/}{MyFigures/chapter1/}}
\else
    \graphicspath{{MyFigures/chapter1/EPS/}{MyFigures/chapter1/}}
\fi

\chapter{緒論}\label{1}


\section{前言}\label{1-1}
希望你能畢業\cite{1},喔不是一定會畢業\cite{2}。

\newpage

\section{研究動機}\label{1-2}
我知道還有很多問題\cite{Bartoletti:2011:UPN:2093698.2093873}\cite{7410059}\cite{5466205}\cite{8103771}與\cite{1010}\cite{8262565}...等\cite{11},哈哈哈\cite{1008}不過一定能解決


\section{論文架構}\label{1-3}
\n 本論文編排方式如下:

第\ref{2}章 說明本研究平台的硬體配備說明,並介紹系統原理與平台架構

第\ref{3}章 說明系統架構與操作

第\ref{4}章 驗證系統的結果
		
         驗證系統之結果

         驗證系統1與系統2整合之結果

第\ref{5}章 結論與未來展望


\chapter{Conclusions and Future Work} \label{conclusion_and_future}
In the Introduction, we expressed the hope that we could made three distinctive contributions toward WSN, WRSN and FRID network, respectively. In this Chapter, we will conclude by describing the proposed method process.
We will also suggest some future work that could provide general view about proposed practically applications
in this thesis.   

\section{Conclusions}

Firstly, in Chapter \ref{chapter1}, we investigated the weighted-critical-square-grid
coverage problem, which is the problem of using limited sensors to
construct a wireless sensor network such that the total weight of
the covered critical square grids is maximized. The problem was
shown to be NP-complete. In addition, a reduction that transforms
the weighted-critical-square-grid coverage problem into the
constrained node-weighted Steiner tree problem was proposed. Once a
solution to the constrained node-weighted Steiner tree problem is
obtained, the solution can be used to select the points that are
allowed to deploy sensors for the weighted-critical-square-grid
coverage problem. We also showed that the constrained node-weighted
Steiner tree problem is NP-complete. In addition, the greedy
algorithm (GA), the group-based algorithm (GBA), and the
profit-based algorithm (PBA), were proposed for the constrained
node-weighted Steiner tree problem.

In Chapter \ref{chapter2}, we then presented the problem of scheduling minimum mobile
devices to periodically recharge and collect data from sensors
subject to the limited charging range, electric capacity, and memory
storage, such that the network lifetime can be guaranteed to be
prolonged without limits, termed the Periodic Energy Replenishment
and Data Collection (PERDC) problem. We first showed
that the PERDC problem is NP-complete. In addition, three
algorithms, including the grid-based algorithm (GBA), the
dominating-set-based algorithm (DSBA), and the
circle-intersection-based algorithm (CIBA), were proposed to find a
set of anchor points. Based on the generated anchor points, the
mobile device scheduling algorithm (MDSA) was proposed to schedule
minimum mobile devices for energy replenishment and data collection.
Moreover, theoretical analysis showed that the GBA had the lowest
time complexity compared with the DSBA and the CIBA when the side
length of the sensor field and the sensor's charging range are
constant. In the future works, we wish to evaluate some metrics
of the network to show the performance of our proposed via extensive simulations.

Finally, in Chapter \ref{chapter3}, we introduced 
the problem of activating readers and
adjusting their interrogation ranges to cover maximum tags without
collisions, termed the Reader-Coverage Collision Avoidance
Arrangement (RCCAA) problem. In order to show the difficulty of the
RCCAA problem, a sub-problem of the RCCAA problem, termed the
Reduced Reader-Coverage Collision Avoidance Arrangement (RRCCAA)
problem, was shown to be NP-hard. In addition, an approximation
algorithm, termed the Maximum-Weight-Independent-Set-Based Algorithm
(MWISBA), was proposed for the RCCAA problem. We also provided
theoretical analysis of our proposed method in terms of the
correctness, the time complexity, and the performance guarantee.
To evaluate our performance, a method that extended from the greedy
distributed elimination (GDE) method \cite{related1}, called the
Extended Greedy Distributed Elimination Algorithm (EGDEA), was
proposed for the RCCAA problem. The MWISBA was compared with the
EGDEA in terms of the number of tags read by readers and the number
of activated readers. The simulation results showed that the readers
activated by the MWISBA can read a significantly higher number of
tags than that in the EGDEA.

\section{Suggestions for Future Work} \label{futurework}

Nowadays, a network model is not simply a static network model when both hardware
and software of network system has been significantly
improved. The designing and proposition of an optimum
solution for network system therefore has
become complicated as well. For example, network
architecture keeps changing constantly and it is divided
into layers with different levels of importance
rather than having static nature as before. More
complicated network models require new techniques
and algorithm to meet real requirements. In addition,
diversified development of functions of network
elements also means more complicated potential conflicts 
in network routing and interfacing between
network elements. To incorporate these functions
and avoid these new conflicts, new network models of
high degree of diversification are required. Based on
experience in the design and optimizing algorithm in
such network types as wireless sensor network, mobile
ad hoc network, RFID system, I will continue to
expand my researches to tackle new issues in different
network architecture.

My current task is to find solutions for developing
and optimizing algorithms in network structure development,
optimization of routing of such networks
as wireless sensor network, mobile ad hoc network,
and RFID system. Nevertheless, to complete and
make my research more independent, I shall continually
expand with new research fields as mentioned to
tackle many aforementioned issues such as complex network, interdependent network. 




% 參考文獻

\renewcommand{\bibname}{參考文獻}
\urlstyle{same}
\printbibliography \addchaptertocentry{\bibname}

% 附錄
% \appendix
% \input{Appendices/AppendixA}

%----------------------------------------------------------------------------------------
%	論文內容 結束
%----------------------------------------------------------------------------------------

% 自傳或簡歷 (可有可無)

% 書背 (此欄為空)

%----------------------------------------------------------------------------------------

\clearpage
\end{CJK}
\end{document}
